\documentclass[]{book}
\usepackage{lmodern}
\usepackage{amssymb,amsmath}
\usepackage{ifxetex,ifluatex}
\usepackage{fixltx2e} % provides \textsubscript
\ifnum 0\ifxetex 1\fi\ifluatex 1\fi=0 % if pdftex
  \usepackage[T1]{fontenc}
  \usepackage[utf8]{inputenc}
\else % if luatex or xelatex
  \ifxetex
    \usepackage{mathspec}
  \else
    \usepackage{fontspec}
  \fi
  \defaultfontfeatures{Ligatures=TeX,Scale=MatchLowercase}
\fi
% use upquote if available, for straight quotes in verbatim environments
\IfFileExists{upquote.sty}{\usepackage{upquote}}{}
% use microtype if available
\IfFileExists{microtype.sty}{%
\usepackage{microtype}
\UseMicrotypeSet[protrusion]{basicmath} % disable protrusion for tt fonts
}{}
\usepackage[margin=1in]{geometry}
\usepackage{hyperref}
\hypersetup{unicode=true,
            pdftitle={Vol. 1: Basics in R},
            pdfauthor={Sarah Schwarts \& Tyson Barrett},
            pdfborder={0 0 0},
            breaklinks=true}
\urlstyle{same}  % don't use monospace font for urls
\usepackage{natbib}
\bibliographystyle{apalike}
\usepackage{longtable,booktabs}
\usepackage{graphicx,grffile}
\makeatletter
\def\maxwidth{\ifdim\Gin@nat@width>\linewidth\linewidth\else\Gin@nat@width\fi}
\def\maxheight{\ifdim\Gin@nat@height>\textheight\textheight\else\Gin@nat@height\fi}
\makeatother
% Scale images if necessary, so that they will not overflow the page
% margins by default, and it is still possible to overwrite the defaults
% using explicit options in \includegraphics[width, height, ...]{}
\setkeys{Gin}{width=\maxwidth,height=\maxheight,keepaspectratio}
\IfFileExists{parskip.sty}{%
\usepackage{parskip}
}{% else
\setlength{\parindent}{0pt}
\setlength{\parskip}{6pt plus 2pt minus 1pt}
}
\setlength{\emergencystretch}{3em}  % prevent overfull lines
\providecommand{\tightlist}{%
  \setlength{\itemsep}{0pt}\setlength{\parskip}{0pt}}
\setcounter{secnumdepth}{5}
% Redefines (sub)paragraphs to behave more like sections
\ifx\paragraph\undefined\else
\let\oldparagraph\paragraph
\renewcommand{\paragraph}[1]{\oldparagraph{#1}\mbox{}}
\fi
\ifx\subparagraph\undefined\else
\let\oldsubparagraph\subparagraph
\renewcommand{\subparagraph}[1]{\oldsubparagraph{#1}\mbox{}}
\fi

%%% Use protect on footnotes to avoid problems with footnotes in titles
\let\rmarkdownfootnote\footnote%
\def\footnote{\protect\rmarkdownfootnote}

%%% Change title format to be more compact
\usepackage{titling}

% Create subtitle command for use in maketitle
\newcommand{\subtitle}[1]{
  \posttitle{
    \begin{center}\large#1\end{center}
    }
}

\setlength{\droptitle}{-2em}

  \title{Vol. 1: Basics in R}
    \pretitle{\vspace{\droptitle}\centering\huge}
  \posttitle{\par}
    \author{Sarah Schwarts \& Tyson Barrett}
    \preauthor{\centering\large\emph}
  \postauthor{\par}
      \predate{\centering\large\emph}
  \postdate{\par}
    \date{2018-08-13}

\usepackage{booktabs}
\usepackage{amsthm}
\makeatletter
\def\thm@space@setup{%
  \thm@preskip=8pt plus 2pt minus 4pt
  \thm@postskip=\thm@preskip
}
\makeatother

\begin{document}
\maketitle

{
\setcounter{tocdepth}{1}
\tableofcontents
}
\chapter{Introduction}\label{introduction}

This is the first volume in the eBook encyclopedia.

\chapter{Software Installation}\label{software-installation}

Here is where we talk about installing software.

\begin{center}\rule{0.5\linewidth}{\linethickness}\end{center}

\section{R}\label{r}

\subsection{First Time Installation}\label{first-time-installation}

\subsection{Periotic Updating}\label{periotic-updating}

\begin{center}\rule{0.5\linewidth}{\linethickness}\end{center}

\section{R Studio}\label{r-studio}

\subsection{First Time Installation}\label{first-time-installation-1}

\subsection{Periotic Updating}\label{periotic-updating-1}

\subsection{Panel Layout}\label{panel-layout}

\begin{center}\rule{0.5\linewidth}{\linethickness}\end{center}

\section{TeX (optional)}\label{tex-optional}

\subsection{TinyTeX}\label{tinytex}

\subsection{MAc}\label{mac}

\subsection{Windows}\label{windows}

\chapter{Packages Management}\label{packages-management}

We describe packages and their management

\begin{center}\rule{0.5\linewidth}{\linethickness}\end{center}

\section{What are packages}\label{what-are-packages}

\begin{center}\rule{0.5\linewidth}{\linethickness}\end{center}

\section{How to install packages}\label{how-to-install-packages}

\begin{center}\rule{0.5\linewidth}{\linethickness}\end{center}

\section{Updating packages}\label{updating-packages}

\begin{center}\rule{0.5\linewidth}{\linethickness}\end{center}

\section{Suggested packages}\label{suggested-packages}

\chapter{Data Management}\label{data-management}

How do you get data into R, view and work with in, and then save it for
later use.

\begin{center}\rule{0.5\linewidth}{\linethickness}\end{center}

\section{Importing Data From Various
Formats}\label{importing-data-from-various-formats}

\subsection{Text Format (.csv, tab-delimited,
ect.)}\label{text-format-.csv-tab-delimited-ect.}

Use \texttt{read.csv()}

\subsection{Excel Format (.xls, .xlsx)}\label{excel-format-.xls-.xlsx}

Use \texttt{haven::read.spss()}

\subsection{SPSS Format (.sav)}\label{spss-format-.sav}

Use \texttt{readxl::read.excel()}

\subsection{REDCap (API directly)}\label{redcap-api-directly}

\begin{center}\rule{0.5\linewidth}{\linethickness}\end{center}

\section{Viewing Data Within R
Studio}\label{viewing-data-within-r-studio}

\subsection{The Environment Tab}\label{the-environment-tab}

\subsection{Notebook Display}\label{notebook-display}

\begin{center}\rule{0.5\linewidth}{\linethickness}\end{center}

\section{Saving Data in R Format}\label{saving-data-in-r-format}

Use \texttt{save(...,\ file\ =\ "name.RData")}

\chapter{Data Wrangling}\label{data-wrangling}

\begin{center}\rule{0.5\linewidth}{\linethickness}\end{center}

\section{Subseting Data}\label{subseting-data}

\subsection{Select Variables (columns)}\label{select-variables-columns}

Use \texttt{dplyr::select()}

\subsection{Select Observations (rows)}\label{select-observations-rows}

Use \texttt{dplyr::filter()}

\begin{center}\rule{0.5\linewidth}{\linethickness}\end{center}

\section{Symbol Opporators}\label{symbol-opporators}

\subsection{Logical Opporators}\label{logical-opporators}

\subsubsection{\texorpdfstring{Ineqalities (\texttt{\textless{}},
\texttt{\textgreater{}}, \texttt{\textless{}=},
\texttt{\textless{}=})}{Ineqalities (\textless{}, \textgreater{}, \textless{}=, \textless{}=)}}\label{ineqalities}

\subsubsection{\texorpdfstring{AND ( \texttt{\&}
)}{AND ( \& )}}\label{and}

\subsubsection{\texorpdfstring{OR
(\texttt{\textbar{}})}{OR (\textbar{})}}\label{or}

\subsubsection{\texorpdfstring{Within a List ( \texttt{\%in\%}
)}{Within a List ( \%in\% )}}\label{within-a-list-in}

\subsection{\texorpdfstring{The Assignemnt Opporator (
\texttt{\textless{}-}
)}{The Assignemnt Opporator ( \textless{}- )}}\label{the-assignemnt-opporator--}

\subsection{\texorpdfstring{The Pipe Opporator (
\texttt{\%\textgreater{}\%}
)}{The Pipe Opporator ( \%\textgreater{}\% )}}\label{the-pipe-opporator}

\bibliography{book.bib,packages.bib}


\end{document}
